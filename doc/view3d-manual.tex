\documentclass[10pt]{article}
\usepackage{amsmath}
\usepackage{mathptmx}
\usepackage{graphicx,color}
\usepackage[margin=1in]{geometry}
\usepackage[all]{xy}
\usepackage{nameref}
\usepackage[colorlinks=true,linkcolor=linkblue]{hyperref}
\definecolor{linkblue}{rgb}{0,0,1}
\usepackage{fancyvrb}
\CustomVerbatimEnvironment{commandline}{Verbatim}{xleftmargin=5mm}
\CustomVerbatimEnvironment{vs3file}{Verbatim}{xleftmargin=5mm}
\CustomVerbatimEnvironment{results}{Verbatim}{xleftmargin=5mm}
\begin{document}
\begin{titlepage}
\begin{center}
\begin{Huge}
View3D User Manual\\
\end{Huge}
\vspace{1cm}
\begin{Large}
Version 3.3, Revision 3\\
\end{Large}
\vspace{1cm}
\begin{Large}
\today
\end{Large}
\end{center}
\end{titlepage}
\section{Background}
This document provides a brief description of use of the View3D 3.3 program for
computing view factors of 3-dimensional geometries described in terms of simple
planar polygons – triangles and convex quadrilaterals. It is an extension of
previous work. The algorithms use improved integration and geometric methods 
with special emphasis on the processing of view obstructions. 

The earlier VLITE program [2] attempted to predict the subdivisions necessary
for the view factor numerical integrations, but this process would occasionally
leave relatively large errors in some of the computed values. There was a major
algorithm revision (3.1) involving adaptive integration in March 1999.  The new
program uses adaptive integration to control the calculation to approximately 
the same level of accuracy for all view factors. Once the view factors are
computed several post-processing operations are executed to combine or separate
surfaces, include the effect of surface emissivity (Hottel's ``script-F'' view
factors or ``total interchange areas'' [3]), and slightly adjust the view 
factors to guarantee conservation of energy when they are used for an enclosure.

It is intended that the algorithms eventually become part of more general heat
transfer programs. The source code is in standard C, which should allow
recompilation for almost any operating system. It should also be relatively 
easy to modify the program to make it a subroutine called from other programs.

\noindent
\textbf{History:}\\
View3D 1.0 -- 1986 -- first program\\
View3D 2.0 -- 1991 -- called VLITE for use with HLITE; conversion from
Fortran to C\\
View3D 3.0 -- 1996 -- search for accurate view factors; split into multiple
programs\\
View3D 3.1 -- 1999 -- adaptive integration, see NISTIR 6925\\
View3D 3.2 -- 2002 -- new controls input for compatibility with anticipated
version 4.0\\

\section{Getting View3D}
\subsection{License}
View3D was written while George Walton was employed at NIST, so the program is
in the public domain with the following disclaimer:
\begin{quotation}
\noindent
This software was developed at the National Institute of Standards and 
Technology by employees of the Federal Government in the course of their 
official duties.  Pursuant to title 17 Section 105 of the United States Code 
this software is not subject to copyright protection and is in the public 
domain. These programs are experimental systems. NIST assumes no responsibility
whatsoever for their use by other parties, and makes no guarantees, expressed 
or implied, about its quality, reliability, or any other characteristic.   We
would appreciate acknowledgement if the software is used. This software can be
redistributed and/or modified freely provided that any derivative works bear
some notice that they are derived from it, and any modified versions bear some
notice that they have been modified.
\end{quotation}
\noindent
Please consider placing modified versions in the public domain or available
under an open source license.
\subsection{System Requirements}
The View3D program is a command line tool written in standard C, and is known
to compile on a number of different systems.  Because of differences between
compilers, some system-specific modifications are required.  Further
information is available in the source distribution. The program has been
implemented without a graphic user interface because its only purpose is to
test the computational algorithms.  Most modern desktop or laptop systems
should have sufficient memory and processor power to run the program.

\subsection{Installation}
Installation files may be provided for some systems in the future, but no
installation is required on most systems once it has been compiled. Compilation
instructions are provided in the source distribution.

\section{Using View3D}
All input and output is through simple ASCII text files. For a reminder of the
sequence of file names on the program command line just enter the program name
followed by a space and then a question mark:

\begin{commandline}
View3D ?
\end{commandline}

\noindent
To computes view factors for a three-dimensional geometry, the usage is: 

\begin{commandline}
View3D <input> <output>
\end{commandline}

\noindent
where \verb|<input>| is the name of the 3-D vertex and surface data file (see
section 5) and \verb|<output>| is the name of the view factor output file (see
section 6).  The filenames may also be entered interactively.  If one or both
of the filenames is missing, the user is prompted to input the required 
file(s).

\section{View3D Surfaces}
View3D directly supports convex planar quadrilaterals and planar triangles. 
Nonconvex planar polygons and polygons with more than four vertices are
supported indirectly: the user must first decompose the polygon into a set of
convex quadrilaterals and triangles, which are combined in a postprocessing
step by the program.  Additionally, a number of different embedded surfaces, or
surfaces that are contained (in the same plane and inside the parent polygon)
within a parent or base surface, are supported.   There are a number of 
different fundamental surface types, and each of these types is modified
depending upon whether the surface is to be combined with other surfaces,
whether the surface is base of other surfaces, and whether it is the subsurface
of another surface.

\subsection{Surface Types}
There are five fundamental surface types:
\begin{enumerate}
\item Conventional surface radiating to other surfaces – These are the most
basic surfaces that are to have view factors calculated.  Each surface has an
active and an inactive side.  The orientation of the surface is determined by
application of the right-hand rule to the vertices that define the surface.
The active side is the positive side according to the right-hand rule.
\item Conventional subsurface radiating to other surfaces – These surfaces are
the most basic subsurface.  As with the conventional surface, each subsurface
has an active and inactive side.  A subsurface interacts radiatively with other
surfaces except the base surface, and may be used to represents situations such
as a window in a wall.  View factors between the base and other surfaces are
reduced by the subsurface values during postprocessing.  The base of a
subsurface may also be a subsurface.  The following restrictions apply:
\begin{enumerate}
   \item The subsurface must lie in the plane of its base surface.
   \item The subsurface must face in the same direction as its base surface.
   \item The subsurface must be entirely within its base surface.
   \item The subsurface must be numbered after its base.
\end{enumerate}
\item Masking surface – A masking surface is much like a subsurface except that
it faces toward the base thereby blocking the view from the masked portion of
the base to other surfaces. The view factor from the mask to the base is 1. 
Masking surfaces are subject to the following restrictions:
\begin{enumerate}
   \item The masking surface must lie in the plane of its base surface.
   \item The masking surface must face in the opposite direction as its base
   surface.
   \item The masking surface must lie entirely within its base surface.
   \item The masking surface must have a conventional surface or subsurface as
   its base surface.
\end{enumerate}
\item Null surface – A null surface is a special type of masking surface that
nullifies the covered portion of the base surface. The null surface is removed
from the matrix of radiating surfaces, and the area of the base surface is
reduced by the area of the null surface during post-processing.  Null surfaces
are subject to the same restrictions as masking surfaces.
\item Non-radiating, view obstructing surface – An obstruction surface may
block the view between radiating surfaces but there is no view factor computed
between it and the radiating surfaces.  As with conventional surfaces, these
surfaces have an active and inactive side.  Obstruction surfaces are allowed to
have regular subsurfaces that will participate as a radiating surfaces.  The
purpose of this arrangement it to simplify the view obstruction calculation
when several surfaces can be grouped as a single view obstructing surface.
\end{enumerate}

\subsection{Surface Combination}
Surface combination allows several surfaces to be combined into a single
surface in the final view factors. This is necessary for describing a
non-convex surface to View3D.  The view factors for surfaces that are to be
combined are calculated separately and then combined using view factor algebra.
For example, the wall around a window could be described as four rectangles
which will be combined during post-processing.  Combination reduces the number
of surfaces and causes a renumbering of the surfaces, so the reported number of
view factors will not match the input number of surfaces.  When combining more
than two surfaces, View3D requires that all of the surfaces be combined with
one surface.  Further information is available below in the format specific
documentation.

\section{Input File Format}
The input file for View3D is an ASCII text file with data entries (often one
per line) that describe the case to be calculated.  The extension ``vs3'' is
usually used for these input files, but this not required.  Each line of a
data file contains specific information in a specific order.
\subsection{Input Data Elements}\label{dataels}
Each entry begins with a single character to specify the kind of element.  The
entries making up an element are generally separated by one or more blanks.
Not all elements are required, and the interpretation of some elements depends
on other elements (most particularly the F element).

\begin{itemize}
\item\textbf{(! /)} A `!' or `/' begins a comment which reports information
to the user but is not read by the program. Use comments to document special
features described by the input file. A comment may also begin after the data
on a line.

\item\textbf{(T t)} A `T' or `t' indicates a title. The remaining information
on this line is transferred to all intermediate files and is echoed in all view
factor reports. This is usually the first line of the input file. If there are
multiple title lines, only the contents of the last will be echoed.

\item\textbf{(* E e)} A `*' or `E' or `e' indicates end-of-information which
terminates reading the input file. Any information on the file after this line
will not be read. This allows you to store additional information, such as
different geometric descriptions, at the end of the input file.

\item\textbf{(C c)} The control line specifies white space separated parameters
as name = value pairs.  Any of the following parameters may be included in any
order. For each parameter, the default value is shown in parentheses.
\begin{itemize}
\item\textbf{eps (1.0e-4):} integration convergence criterion for both adaptive
integration and view obstruction calculations.  This is not an exact measure of
the accuracy of the computed view factors, but smaller values will usually lead
to more precise values. The convergence criteria should not be less than about
1.0e-6 because many of the intermediate calculations are accurate only to single
(32-bit) precision.
\item\textbf{maxU (12):} maximum recursions used in computing the unobstructed
view factors.
\item\textbf{maxO (8):} maximum recursions used in computing the obstructed
view factors. Limiting the maximum number of recursions limits the total
execution time of the program but may prevent reaching the specified
convergence.
\item\textbf{minO (0):} minimum recursions used in computing the obstructed
view factors. This can help in cases where an obstruction occurs very near the
view between the edges of two surfaces. The normal adaptive integration may
miss the obstruction. Increasing this value from its normal value of 0 to 1 
or 2 may catch the obstruction. This is probably not necessary except when very
accurate view factors are desired. It can add considerably to execution time.
\item\textbf{row (0):} selected row for computing view factors (0 = all rows)
\item\textbf{col (0):} selected column for computing view factors (0 = all
columns)
\item\textbf{encl (0):} 1 indicates that the surfaces form an enclosure; 0
indicates that they do not. This data is used to adjust the view factors of an
enclosure to guarantee conservation of energy.
\item\textbf{emit (0):} 1 indicates that diffuse reflectance effects will be
included in the computed view factors (Hottel’s “script-F” view factors); 0
indicates they will not, i.e., surfaces will be considered 'black'.
\item\textbf{out (0):} view factor output file format -- 0 = simple text file; 
1 = binary file.
\item\textbf{list (0):} computational summary written to the View3D.log file -- 
0 gives minimal information; 1 gives slightly more; 2 prints all the view
factors; 3 causes dumping of some intermediate values.
\end{itemize}

\item\textbf{(F f)} Most of the input file is devoted to describing the
geometry of the surfaces plus some related surface data.  There are several
ways to describing surface geometry. The geometry line indicates the type of
geometry being described by the rest of the input file. It consists of a `F'
or `f' followed by a single parameter: 3, or 3a (2 is reserved for a 2-D
geometry processed by another program and 1 is reserved for a ``1-D'' geometry
using only surface areas for “very approximate” view factors).


\item\textbf{(V v)} A `V' or `v' indicates a line that describes a vertex.  The
character is followed by a vertex number (a positive integer greater than 0)
and two floating point coordinate values.  This data element is only used for
F=3.  Vertices should be specified in order, as in this example:

\begin{vs3file}
!  #   x    y    z      coordinates of vertices
V  1   0.   0.   0.
V  2   1.   0.   0.
V  3   1.   1.   0.
V  4   0.   1.   0.
\end{vs3file}

\item\textbf{(S s) (M m) (N n) (O o)} Each line beginning with one of these
characters is a surface to be used in the view factor calculation.  The format
of the data depends upon the value of the F element.  See \ref{Feq3} and 
\ref{Feq3a} for details.
\end{itemize}

\subsection{View3D Format (F=3)}\label{Feq3}
This format specifies vertices and surfaces separately.  Control elements are
specified first, then vertices, followed by surfaces.  Each surface is
specified by a surface number (a positive integer greater than one), four
vertex numbers, a base surface number, a combination surface number, an
emissivity, and a surface name.  Surfaces should also be specified in order,
as in the following example:

\begin{vs3file}
!  #   v1  v2  v3  v4 base cmb  emit  name
S  1    1   2   3   4   0   0   0.50  zeq0
S  2    1   4   8   5   0   0   0.50  xeq0
\end{vs3file}

The vertex pointers are the numbers of the previously defined vertices listed 
in counter-clockwise sequence as viewed from the front of the surface. If the
surface is a triangle, the fourth vertex number should be zero. Quadrilateral
surfaces must be convex, i.e. no interior angle greater than 180 degrees.  The
base surface number associates a surface with another surface as a subsurface
of the other subsurface.  In the above example, neither surface is a subsurface
(base = 0).  Subsurfaces should always be specified after the base surface.
The combination surface number (the ``cmb'' column above) specifies that
surfaces are to be combined at the output step.  A non-zero value indicates the
number of the surface with which this surface is combined.  The combination
surface number should always refer to a previously defined surface.  When
combining more than two surfaces, combine all the higher numbered surfaces
with the lowest numbered surface (do not ``chain'' combinations).  The surface
emissivity allows for the modeling of ``gray'' reflection from non-black
surfaces. The values may range between 0.01 (very reflective) and 0.99 (near
black).  The surface name helps to identify surfaces in the reports. A name 
may have no embedded blanks; use `\verb|_|', `\verb|-|', or selective
capitalization instead of blanks.

\begin{vs3file}
Example of a F=3 input file:
T  Pinney & Bean test 4: L shaped room (page A2.14; Table A2.11) BRE
! encl list  epsu  epso  maxu maxo mino emit
C  encl=1 list=2 eps=1.e-4 maxu=8 maxo=8 mino=0 emit=1
F  3
!  #   x    y    z      coordinates of vertices
V  1   0.   3.   3.
V  2   1.   3.   3.
V  3   0.   3.   0.
V  4   1.   3.   0.
V  5   0.   1.   0.
V  6   1.   1.   0.
V  7   3.   1.   0.
V  8   0.   0.   0.
V  9   1.   0.   0.
V 10   3.   0.   0.
V 11   0.   0.   3.
V 12   1.   0.   3.
V 13   3.   0.   3.
V 14   0.   1.   3.
V 15   1.   1.   3.
V 16   3.   1.   3.
!  #   v1  v2  v3  v4 base cmb  emit  name      surface data
S  1   11  13  10   8   0   0   0.90  srf-1
S  2   10  13  16   7   0   0   0.90  srf-2
S  3    6   7  16  15   0   0   0.90  srf-3
S  4    4   6  15   2   0   0   0.90  srf-4
S  5    3   4   2   1   0   0   0.90  srf-5
S  6    8   3   1  11   0   0   0.90  srf-6
S  7   11  14  15  12   0   0   0.90  srf-7
S  8    8   9   6   5   0   0   0.90  srf-8
S  9    1   2  15  14   0   7   0.90  srf-7b    ! surfaces that combine
S 10   15  16  13  12   0   7   0.90  srf-7c
S 11    5   6   4   3   0   8   0.90  srf-8b
S 12    9  10   7   6   0   8   0.90  srf-8c
End of data
\end{vs3file}

\subsection{Blast Surface Format (F=3a)}\label{Feq3a}
This format specifies vertices and surfaces simultaneously, and specifies more
information about the surface shape.  As with the F=3 format, control elements
are specified first followed by surfaces.  Each surface is then described by
two or three lines of data which fully describe the surface and its vertices.

The first line is similar to the surface lines described above except that
instead of four vertex pointers there is a shape indicator: R for a rectangle,
Q for a general quadrilateral, and T for a triangle.  The data sequence is:
surface type (S, M, N, or O), surface shape (R, Q or T), a possible base
surface number, a possible surface combination number, the emissivity of the
surface, and the surface name. The surfaces must be numbered in order beginning
at 1.  The second line will include the 3-D coordinates (X, Y, Z) of the origin
a plane containing the surface, the azimuth angle (clockwise from the Y axis -
north) of that plane, and its tilt angle (vertical = 90, facing flat upward = 0,
flat downward = 180).  The remainder of data depends upon the surface type, 
as in this example:

\begin{vs3file}
!  #  s base cmb emit  name
S  1  R   0   0  0.90  srf-1
  3. 0. 0.    0  90    3. 3.
\end{vs3file}

A rectangle (shape = R) is described by its width and height.  The origin
coordinates refer to the lower left corner of the rectangle as viewed from the
'front' or active side.  The above example is a 3 by 4 square.  The vertices of
a quadrilateral (or triangle) is described by four (or 3) sets of (X, Y)
coordinates relative to the origin of the plane.  These additional data values
may be placed upon a separate line.

\begin{vs3file}
Example of a F=3a input file:
T Pinney & Bean test 4: L shaped room (page A2.14; Table A2.11) BRE
C  encl=1 list=2 eps=1.e-5
F  3a
!  #  s base cmb emit  name
S  1  R   0   0  0.90  srf-1
  3. 0. 0.    0  90    3. 3.
S  2  R   0   0  0.90  srf-2
  3. 1. 0.  270  90    1. 3.
S  3  r   0   0  0.90  srf-3
  1. 1. 0.  180  90    2. 3.
S  4  R   0   0  0.90  srf-4
  1. 3. 0.  270  90    2. 3.
S  5  R   0   0  0.90  srf-5
  0. 3. 0.  180  90    1. 3.
S  6  R   0   0  0.90  srf-6
  0. 0. 0.   90  90    3. 3.
S  7  Q   0   0  0.90  srf-7
  3. 0. 3.    0  180 
  2. 0.   3. 0.   3. 1.   2. 1.
S  8  Q   0   0  0.90  srf-8
  0. 0. 0.  180   0 
  0. 0.   1. 0.   1. 1.   0. 1.
S  9  Q   0   7  0.90  srf-7b
  3. 0. 3.    0  180 
  2. 1.   3. 1.   3. 3.   2. 3.
S 10  Q   0   7  0.90  srf-7c
  3. 0. 3.    0  180 
  0. 0.   2. 0.   2. 1.   0. 1.
S 11  Q   0   8  0.90  srf-8b
  0. 0. 0.  180.  0.
  0. 1.   1. 1.   1. 3.   0. 3.
S 12  q   0   8  0.90  srf-8c
  0. 0. 0.  180.  0.
  1. 0.   3. 0.   3. 1.   1. 1.
End of data
\end{vs3file}

\section{Program Output}
View3D writes data in three forms: general status information to the standard
output, more detailed status information to a log file, and final view factors
to an output file.  The output to the standard output is self-explanatory and
not very detailed, so is not covered here.  Both the log file and output file
contain more information.  For demonstration purposes, the view factors for a
simple cube with unity side lengths is computed.

\begin{vs3file}
Input file for a cube:
T A simple cube
C encl=1 list=3
F 3
!  #   x    y    z      coordinates of vertices
V  1   0.   0.   0.
V  2   1.   0.   0.
V  3   1.   1.   0.
V  4   0.   1.   0.
V  5   0.   0.   1.
V  6   1.   0.   1.
V  7   1.   1.   1.
V  8   0.   1.   1.
!  #   v1  v2  v3  v4 base cmb  emit  name      surface data
S  1    1   2   3   4   0   0   0.50  zeq0
S  2    1   4   8   5   0   0   0.50  xeq0
S  3    1   5   6   2   0   0   0.50  yeq0
S  4    7   6   5   8   0   0   0.50  zeq1
S  5    7   3   2   6   0   0   0.50  xeq1
S  6    7   8   4   3   0   0   0.50  yeq1
End of data
\end{vs3file}

\subsection{View Factor Output File}
The first line of the output file contains a header that provides information
on the case that has been processed:

\begin{results}
View3D version out encl emit N
\end{results}

\noindent
The version entry is the version of View3D that was used, N is the number of
surfaces described in this file, and the remainder are the control parameters
of the same names.  It is important to note that if surface combination has
been requested, then the surface count reported here will not be the same as
the number of input surfaces.  

The second line lists the surface areas in order for each output surface.  
Again, if surface combinations have taken place, then the areas of the surfaces
reported will reflect the combinations.

The next N lines (for N surfaces) are a square matrix of view factors.  If
emittances have not been included (emit=0, which is the default), then the
diagonal of this matrix will be zero.  If emittances have been included 
(emit=1), then the output matrix contains Hottel’s ``script-F'' view factors
[3]. These factors may be used with the ViewHT program to solve the heat
transfer problem.  The view factors are arranged logically: the view factor
 between surface $m$ and surface $n$ is entry $n$ in row $m$.

The final line of the file lists the surface emittances, again in order and
again reflecting any combinations of surfaces.

The complete output file for the cube case is given below.  Note that in this
case the view factor matrix is symmetric.  

\begin{results}
Example output file:
View3D 3.3.1 0 1 0 6          
1 1 1 1 1 1
0.000000 0.200044 0.200044 0.199825 0.200044 0.200044
0.200044 0.000000 0.200044 0.200044 0.199825 0.200044
0.200044 0.200044 0.000000 0.200044 0.200044 0.199825
0.199825 0.200044 0.200044 0.000000 0.200044 0.200044
0.200044 0.199825 0.200044 0.200044 0.000000 0.200044
0.200044 0.200044 0.199825 0.200044 0.200044 0.000000
0.500 0.500 0.500 0.500 0.500 0.500
\end{results}

These results are easy to verify.  Formulas for the view factor between
opposite sides of the cube are available (e.g. Equation 2-30 in [3]), gives

\begin{displaymath}
F_{14}=0.199824896.
\end{displaymath}

The remaining factors for surface 1 may be computed with view factor algebra.
Since the cube is an enclosure, the factors for surface 1 must sum to one.
The factors for all the faces perpendicular faces must be the same by symmetry,
so

\begin{displaymath}
F_{12}=F_{13}=F_{15}=F_{16}=\frac{1}{4}(1-F_{14})=0.200043776.
\end{displaymath}

\noindent
These results agree with the View3D calculations.

\subsection{View3D Log File}
A log file (View3D.log) is produced whenever View3D is run. This file is
written in the directory containing the executable.  It provides information
on the calculation of the view factors.  The input is summarized and the
progress of the calculation is updated.  The extent of the data that is written
is determined by the list input element described in section \ref{dataels} above.

\section{References}
[1] Walton, G.N., ``Algorithms for Calculating Radiation View Factors Between
Plane Convex Polygons with Obstructions'', National Bureau of Standards NBSIR
86-3463, Gaithersburg MD (1986)

\noindent
[2] Walton, G.N., ``Computer Programs for Simulation of Lighting/HVAC
Interactions'', National Institute of Standards and Technology NISTIR 5322,
Gaithersburg MD (1993)

\noindent
[3] Hottel, H.C. and A.F. Sarofim, Radiative Transfer, McGraw Hill, New York 
NY (1967)

\noindent
[4] Pinney, A.A. \& M.W. Bean, ``A Set of Analytic Tests for Internal Longwave
Radiation and View Factor Calculations'', Vol II, An investigation into 
Analytical and Empirical Validation Techniques for Dynamic Thermal Models of
Buildings, BRE, Garston, Watford, WD2 7JR, UK (1988)

\noindent
[5] Shapiro, A.B., ``FACET – A Radiation View Factor Computer Code for
Axisymmetric, 2D Planar, and 3D Geometries with Shadowing'', University of
California, Lawrence Livermore National Laboratory, UCID-19887 (1983)

\end{document}
